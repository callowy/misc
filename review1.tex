\documentclass[11pt]{beamer}
\usepackage[utf8]{inputenc}
\usepackage[T1]{fontenc}
\usepackage{lmodern}
\usetheme{default}
\usepackage{amsmath}
\usepackage{amssymb}
\usetheme{Madrid}
\usecolortheme{seahorse}

\newcommand*\colvec[3][]{
	\begin{pmatrix}\ifx\relax#1\relax\else#1\\\fi#2\\#3\end{pmatrix}
}

\begin{document}
	\author{Colin Moore}
	\title{Linear Algebra Review 1}
	\subtitle{MAT 442}
	%\logo{}
	%\institute{}
	\date{Nov 2020}
	\subject{MAT 442}
	%\setbeamercovered{transparent}
	%\setbeamertemplate{navigation symbols}{}
	\begin{frame}[plain]
		\maketitle
	\end{frame}

\begin{frame}
	\frametitle{Vector Spaces}
	%\framesubtitle{subtitle}
	A vector space V over a field F has the following properties:
	\begin{enumerate}
		\setbeamertemplate{enumerate items}[default]
		\item $\forall x, y \in V$, $x + y = y + x$
		\item $\forall x, y, z \in V$, $(x + y) + z = x + (y + z)$
		\item $There \ exists\ an\ element\ in\ V,\ denoted\ by\ 0,\ such\ that\ \forall x \in V, x + 0 = x$
		\item $\forall x \in V, \: \exists y \in V\ such\ that\ x + y = 0$
		\item $\forall x \in V, \: 1\cdot x = x$
		\item $\forall \alpha, \beta \in F\ and\ \forall x \in V, (\alpha\beta)x = \alpha(\beta x)$
		\item $\forall \alpha \in F\ and\ \forall x, y \in V,\ 
				\alpha(x+y) = \alpha x + \alpha y$
		\item $\forall \alpha, \beta \in F,\ and\ \forall x \in V,\ 
				(\alpha + \beta)x = \alpha x + \beta x$
	\end{enumerate}

	\begin{block}{Remark}
		Vector multiplication need not be defined. Only \alert{scalar} multiplication and vector addition are required.
	\end{block}
\end{frame}

\begin{frame}
	\frametitle{Vector Spaces}
	\begin{block}{Thereom}
		Let V be a vector space and W a subset of V. Then W is a subspace of V if and only if:
		\begin{enumerate}
			\setbeamertemplate{enumerate items}[default]
			\item $0 \in W$
			\item $x + y \in W\ whenever \ x, y \in W$
			\item $\alpha x \in W\ whenever\ \alpha \in F\ and\  x \in W$
		\end{enumerate}
	\end{block}

	\begin{itemize}
		\item Is the intersection of subspaces of a vector space V a subspace of V?
		\item Is the union of subspaces of a vector space V a subspace of V?
	\end{itemize}
\end{frame}

\begin{frame}
	\frametitle{Linear Transformations}
	\textbf{Definiton:} Let V and W be vector spaces over F. We call a function $T: V \rightarrow W $ a \textbf{linear transformation from V to W} if, $\forall x,y \in V$ and $\forall \alpha \in F$, we have:
	\begin{enumerate}
		\setbeamertemplate{enumerate items}[default]
		\item T$(x + y)$ = T$(x)$ + T$(y)$
		\item T$(\alpha x)$ = $\alpha$T$(x)$
	\end{enumerate}
\phantom{}


$\bullet$ If T is linear, then T$(0)$ = $0$ \\
$\bullet$ To show that a given transformation is linear, it is enough to prove that T$(\alpha x + y)$ = $\alpha$T$(x)$ + T$(y)$

\phantom{}

\textbf{Problem:} Is the following transformation T: $V \rightarrow V$  linear?
\[  T(A) = A^{t} \ with \ F = \mathbb{C}, V = Mat_{2}(\mathbb{C}) \]
\end{frame}

\begin{frame}
	\frametitle{Linear Transformations}
	\textbf{Definition:} Let V and W be vector spaces and let T: V $\rightarrow$ V be linear. We define the \textbf{null space} (or \textbf{kernel}) of T as:
	\[ N(T) = \{x \in V: T(x) = 0 \}\]
	We define the \textbf{range} (or \textbf{image}) of T as:
	\[ R(T) = \{T(x) : x \in V \}\]
	or in other words:
	\[ R(T) = \{y \in W : T(x) = y \}\]	
	
Important note: N(T) is a subspace of \textbf{V} while R(T) is a subspace of \textbf{W}

\phantom{}

\textbf{Problem:} Find N(T) and R(T) for a linear transformation T: R\textsuperscript{3} $\rightarrow$ R\textsuperscript{2} defined by
\[T(a_{1}, a_{2}, a_{3}) = (a_{1} - a_{2}, 2 a_3) \]
\end{frame}

\begin{frame}
	\frametitle{Linear Transformations}
	\begin{block}{Theorem}
		Let V and W be vector spaces, and let T: V $\rightarrow$ W be linear. If $\beta = \{v_{1}, v_{2}, ..., v_{n}\}$ is a basis for V, then
		\[ R(T) = \text{span}(T(\beta)) = \text{span}(\{T(v_{1}), T(v_{2}), ..., T(v_{n})\}) \]
	\end{block}

\textbf{Example:} Let T: R\textsuperscript{3} $\rightarrow$ R\textsuperscript{3} be defined by T(a, b, c) = (a+b, 2b, 0). Find a basis for R(T). \\

\phantom{}

Using the standard basis for R\textsuperscript{3}, we see

	\begin{align*}
		R(T) &= \text{span}(\{ T(1,0,0), T(0,1,0), T(0,0,1) \}) \\
			&= \text{span}(\{(1,0,0), (1,2,0), (0,0,0) \}) \\
			&= \text{span}(\{(1,0,0), (1,2,0) \})
	\end{align*}
\end{frame}

\begin{frame}
	\frametitle{Linear Transformations}
	\begin{block}{Dimension Theorem}
		Let V and W be vector spaces, and let T: V $\rightarrow$ W be linear. If V is \alert{finite-dimensional}, then
		\[\text{nullity(T) + rank(T) = dim(V)} \]
	\end{block}

Proof:
\begin{itemize}
	\setbeamertemplate{itemize items}[default]
	\item Suppose dim(V) = n, dim(N(T)) = k, define a basis for N(T) as $\{v_{1}, ..., v_{k}\}$
	\item Extend to a basis for V, so $\beta = \{v_{1}, ..., v_{k}, v_{k+1}, ..., v_{n} \}$ 
	\item Claim $S = \{T(v_{k+1}),  T(v_{k+2}), ..., T(v_{n})\}$ is a basis for R(T)
	\item $T(v_{i}) = 0$ for $1 \leq i \leq k$ since $v_{i}$ comes from the null space of T, thus
		\begin{align*}
			R(T) &= \text{span}(T(\beta)) = \text{span}(\{T(v_{1}), ..., T(v_{n})\}) \\
			&= \text{span}( \{ 0, ..., T(v_{k+1}), T(v_{k+2}), ..., T(v_{n}) \} ) \\
			&= \text{span}( 
			\{ T(v_{k+1}), T(v_{k+2}), ... , T(v_{n}) \} )
		\end{align*}
\end{itemize}
\end{frame}

\begin{frame}
	\frametitle{Linear Transformations}
	Proof (cont):
	\begin{itemize}
		\setbeamertemplate{itemize items}[default]
		\item Correct number of vectors, need to check for linear independence. Suppose we have
		$b_{k+1}T(v_{k+1}) + b_{k+2}T(v_{k+2}) + ... + b_{n}T(v_{n}) = 0$
		\item T is linear, thus $T\left( b_{k+1}v_{k+1} + b_{k+2}v_{k+2} + ... b_{n}v_{n} \right) = 0 $
		\item Hence $ b_{k+1}v_{k+1} + b_{k+2}v_{k+2} + ... b_{n}v_{n} \in N(T) $
		\item $\{v_{1}, v_{2}, ... , v_{k}\}$ is a basis for N(T), so $\exists \mu_{1},...\mu_{k} \in F$ such that $b_{k+1}v_{k+1} + ... + b_{n}v_{n} = \mu_{1}v_{1} + ... + \mu_{k}v_{k}$
		\item Rewrite as $\mu_{1}v_{1} + ... + \mu_{k}v_{k} - b_{k+1}v_{k+1} - ... - b_{n}v_{n} = 0$
		\item But notice that $\{v_{1}, ..., v_{k}, v_{k+1}, ..., v_{n} \} = \beta $, a basis for V. Thus $b_{i} = 0$ for all i. Thus S is linearly independent.
		\item S spans R(T) and is linearly independent, thus it is a basis for R(T)
		\item Therefore rank T = k - n
	\end{itemize}
\end{frame}

\begin{frame}
	\frametitle{Linear Transformations}
	\textbf{Definition:} Let $\beta = \{u_{1}, u_{2}, ..., u_{n}\}$ be an ordered basis for a finite dimensional vector space V. For $x \in V$ we can represent x as a linear combination of $\beta$ with $a_{i} \in F$:
	\[x = a_{1}u_{1} + a_{2}u_{2} + ... + a_{n}u_{n}\]
	We define the \textbf{coordinate vector of $x$ relative to $\beta$}, denoted $\left[x\right]_{\beta}$, by
	\[ \left[x\right]_{\beta} = (a_{1}, a_{2}, ..., a_{n})^{t}\]
	
	\begin{block}{Definition}
		We call the m$\times$n matrix A defined by $A_{ij} = a_{ij}$ the \textbf{matrix representation of T in the ordered bases $\beta$ and $\gamma$} and write A = $\left[ T \right]^{\gamma}_{\beta}$
	\end{block}

Note that the \textit{j}th column of A is $\left[T(v_{j})\right]_{\beta}$
\end{frame}

\begin{frame}
	\frametitle{Linear Transformations}
	\textbf{Example: } Let T: R\textsuperscript{2} $\rightarrow$ R\textsuperscript{3} be the linear transformation defined by
	\[ T(a_{1}, a_{2}) = (0,\ 2a_{1} + 2a_{2},\ a_{1} - 3a_{2})\]
	Let $\beta$ and $\gamma$ be the standard ordered bases for R\textsuperscript{2} and R\textsuperscript{3} respectively
	\[ T(1,0) = (0,2,1)\]
	\[ T(0,1) = (0, 2, -3)\]
	Thus the matrix representation of T is
	\[ \left[T \right]_{\beta}^{\gamma}\ = \begin{pmatrix}
		0 & 0 \\
		2 & 2 \\
		1 & -3
	\end{pmatrix}\] 
\end{frame}

\begin{frame}
	\frametitle{Linear Transformation}
	\textbf{Problem: }Let T: R\textsuperscript{2} $\rightarrow$ R\textsuperscript{3} be given by T$(x,y) = (x-y,\ y,\ 2x+y)$ 
	$\text{Let } A = \{ (1,2), (2,3)\} $ and $B = \{(1,1,0), (0,1,1), (2,2,3) \}$. Find 
	\begin{itemize}
		\item 	$\left[T \right]_{A}^{B}$
		\item a basis for ker(T)
		\item a basis for range(T)
	\end{itemize}

\end{frame}

\begin{frame}
	\frametitle{Linear Transformation}
	\begin{block}{Theorem}
		Let V and W be finite-dimensional vector spaces having ordered bases $\beta$ and $\gamma$, respectively, and let T: V $\rightarrow$ W be linear. Then, for each $u \in V$, we have
		\[ \left[ T(u) \right]_{\gamma} = \left[ T \right]_{\beta}^{\gamma} \left[ u \right]_{\beta} \]
	\end{block} 
\end{frame}

\begin{frame}
	\frametitle{Linear Transformation}
	\textbf{Example: } Let T: P\textsubscript{3}($ \mathbb{R} $) $\rightarrow$ P\textsubscript{2}($ \mathbb{R} $) be the linear transformation defined by $T(f(x)) = f'(x)$. Let $\beta = \{1, x, x^2, x^3\}$ and let $\gamma = \{1, x, x^2\}$. Then
	\[A = \left[T \right]_{\beta}^{\gamma}  =  
	\begin{pmatrix}
		0 & 1 & 0 & 0 \\
		0 & 0 & 2 & 0 \\
		0 & 0 & 0 & 3
	\end{pmatrix} \]
Let $u(x) = 4 + x + 3x^2 + x^3$. Then
	\[ \left[T(u) \right]_{\gamma}  = \left[ 1 + 6x + 3x^2 \right]_{\gamma} = (1,6,3)^{t} \]
	But also note
	\[ \left[ T \right]_{\beta}^{\gamma} \left[ u \right]_{\beta}  = 
	\begin{pmatrix}
		0 & 1 & 0 & 0 \\
		0 & 0 & 2 & 0 \\
		0 & 0 & 0 & 3
	\end{pmatrix}
	\begin{pmatrix}
		4 \\
		1 \\
		3 \\
		1
	\end{pmatrix} = 
	\begin{pmatrix}
		1 \\
		6 \\
		3
	\end{pmatrix}
\]
\end{frame}

\begin{frame}
	\frametitle{Invertibility}
	\begin{block}{Theorem}
		Let V and W be \textit{finite-dimensional} vector spaces with ordered bases $\beta$ and $\gamma$, respectively. Let T: V $\rightarrow$ W be linear. Then T is invertible if and only if $\left[T\right]_{\beta}^{\gamma}$ is invertible. Furthermore, $\left[T^{-1}\right]_{\gamma}^{\beta} = (\left[ T\right]_{\beta}^{\gamma})^{-1}$
	\end{block}

	\phantom{}
	
	\textbf{Problem:} Let $T: P_{1}(R) \rightarrow R^{2}$ be the linear transformation defined by $T(a + bx) = (a, a+b)$. Given $\beta, \gamma$ to be the standard ordered bases for $P_{1}(R) \ and \ R^{2}$ respectively, find $T^{-1}, \left[ T\right]_{\beta}^{\gamma}, \text{and} \left[ T^{-1} \right]_{\gamma}^{\beta}$.
\end{frame}

\begin{frame}
	\frametitle{Quotient Spaces}
	\begin{block}{Definition}
		Let W be a subspace of a vector space V. For any $v \in V$, the set $\{v\} + W = \{v + w: w \in W\}$ is called the \textbf{coset} of $W$ \textbf{containing} $v$. We define the \textbf{quotient space} of $V$ modulo $W$ using the following operations:
		\begin{align*}
			(x + W) + (y + W) &= (x + y) + W \\
			\alpha (x + W) &= (\alpha x) + W
		\end{align*}
	\end{block}
	
	\phantom{}
	
	\textbf{Problem:} Show that $x + W = y + W$ if and only if $x-y \in W$
	
	\phantom{}

	\textbf{Problem: } Let V = $\{a_{0} + a_{1}X + ... + a_{3}X^{3} : a_{j} \in \mathbb{Q}\}$, a vector space over $\mathbb{Q}$ under the usual polynomial addition and scalar multiplication. Let W = span$( \{x+1, 2x-1, x^{2}+x\})$. Find a spanning set for $V/W$.
\end{frame}

\end{document}